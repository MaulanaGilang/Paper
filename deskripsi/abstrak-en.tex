% Mengubah keterangan `Abstract` ke bahasa indonesia.
% Hapus bagian ini untuk mengembalikan ke format awal.
% \renewcommand\abstractname{Abstrak}

\begin{abstract}

  % Ubah paragraf berikut sesuai dengan abstrak dari penelitian.
  
Concrete plays a crucial role in various infrastructure projects. According to data, about 70\% of construction projects worldwide use concrete as their primary material. The presence of air voids can reduce the strength of concrete by up to 30\% and affect its performance over the long term. In this study, a 2D CNN is used to classify B-Scan signal data generated from GPR simulations using gprMax. The input files for the simulations are automatically prepared using Python, and later generated using a GPU. The data undergoes preprocessing to facilitate the training process. Five 2D CNN models are tested in this experiment, with the first model undergoing three trials to determine the optimal training data split. The trained models are then tested on data prepared for testing. Roboflow is also used to compare with the 2D CNNs, and YOLOv9 is employed as the final stage tester in the website-based detection implementation. The results show that the 2D CNN models can classify B-Scan signal data with an accuracy of 0.9964 in the first model with a data split of 70/15/15. In the validation data, an inference time of 115ms/step and an F1 score of 1.0 are recorded. The classification results with Roboflow also show good accuracy at 0.986. In the final stage testing, YOLOv9 effectively detects air voids with an accuracy of 0.979.


\end{abstract}

% Mengubah keterangan `Index terms` ke bahasa indonesia.
% Hapus bagian ini untuk mengembalikan ke format awal.
% \renewcommand\IEEEkeywordsname{Kata kunci}

\begin{IEEEkeywords}

  % Ubah kata-kata berikut sesuai dengan kata kunci dari penelitian.
  Concrete, Air Gap, CNN, GPR, gprMax.

\end{IEEEkeywords}
